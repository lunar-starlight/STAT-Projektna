\documentclass[a4paper]{article}

% LTeX: enabled=false
% Kodiranje in podpora slovenščini
\usepackage[T1]{fontenc}        % kodiranje znakov v .pdf
\usepackage[slovene]{babel}

\usepackage{fontspec}
\usepackage{lualatex-math}
\usepackage{unicode-math}

% \setmainfont{TeX Gyre Pagella}
% \setmathfont{TeX Gyre Pagella Math}
\setmathfont{Latin Modern Math}
\setmathfont{Asana Math}[range={scr}]
\setmathfont{STIX Two Math-Regular}[range={bb}]
% \setmathfont{Asana Math}[range={"007B,"007D}]  % {}
\setmathfont{Asana Math}[range={8709, \setminus}]  % U+2205, emptyset


% \usepackage{concrete}           % pisave po okusu Donalda Knutha

% Ta paket potrebujemo, ker amsart spreminja male črke v velike v naslovu,
% in tega ne zna pravilno delati s šumniki. Paket textcase ta problem odpravi.
% Glej: https://tex.stackexchange.com/questions/211305/problem-with-greek-title
\usepackage{textcase}

% Paketi za matematiko

\undef\eth
\undef\digamma
\undef\backepsilon
\usepackage{amsmath}  % razna okolja za poravnane enačbe ipd.
\usepackage{amsthm}   % definicije okolij za izreke, definicije, ...
% \usepackage{xypic}    % paket za diagrame

% \usepackage{mathtools}


\usepackage{fancyhdr}
\usepackage{extramarks}
\usepackage{enumerate}
\usepackage{tikz}
\usetikzlibrary{babel}
% \usepackage[plain]{algorithm}
% \usepackage{algpseudocode}
% \usepackage{listings}
\usepackage{minted}
\usepackage{hyperref}

\usetikzlibrary{cd,positioning}

%
% Basic Document Settings
%

\topmargin=-0.45in
\evensidemargin=0in
\oddsidemargin=0in
\textwidth=6.5in
\textheight=9.0in
\headsep=0.25in
\headheight=12.5pt

\linespread{1.1}

\pagestyle{fancy}
\lhead{\hmwkAuthorName}
\chead{\hmwkClass:\ \hmwkTitle}
\rhead{\firstxmark}
\lfoot{\lastxmark}
\cfoot{\thepage}

\renewcommand\headrulewidth{0.4pt}
\renewcommand\footrulewidth{0.4pt}

\setlength\parindent{0pt}

%
% Create Problem Sections
%

\newcommand{\enterProblemHeader}[1]{
    \nobreak\extramarks{}{Naloga \arabic{#1} se nadaljuje na naslednji strani\ldots}\nobreak{}
    \nobreak\extramarks{Naloga \arabic{#1} (nadaljevanje)}{Naloga \arabic{#1} se nadaljuje na naslednji strani\ldots}\nobreak{}
}

\newcommand{\exitProblemHeader}[1]{
    \nobreak\extramarks{Naloga \arabic{#1} (nadaljevanje)}{Naloga \arabic{#1} se nadaljuje na naslednji strani\ldots}\nobreak{}
    \stepcounter{#1}
    \nobreak\extramarks{Naloga \arabic{#1}}{}\nobreak{}
}

\setcounter{secnumdepth}{0}
\newcounter{partCounter}
\newcounter{solCounter}
\newcounter{homeworkProblemCounter}
\setcounter{homeworkProblemCounter}{1}
\nobreak\extramarks{Naloga \arabic{homeworkProblemCounter}}{}\nobreak{}

%
% Homework Problem Environment
%
% This environment takes an optional argument. When given, it will adjust the
% problem counter. This is useful for when the problems given for your
% assignment aren't sequential. See the last 3 problems of this template for an
% example.
%
\newenvironment{homeworkProblem}[1][-1]{
    \ifnum#1>0
        \setcounter{homeworkProblemCounter}{#1}
    \fi
    \section{Naloga \arabic{homeworkProblemCounter}}
    \setcounter{partCounter}{1}
    \setcounter{solCounter}{1}
    \enterProblemHeader{homeworkProblemCounter}
}{
    \exitProblemHeader{homeworkProblemCounter}
}


%
% Title Page
%

\title{
    \vspace{2in}
    \textmd{\textbf{\hmwkClass:\ \hmwkTitle}}\\
    \normalsize\vspace{0.1in}\small{Rok\ oddaje\ \hmwkDueDate}\\
    % \vspace{0.1in}\large{\textit{\hmwkClassInstructor\ \hmwkClassTime}}
    \vspace{3in}
}

\author{\hmwkAuthorName}
\date{}

% \renewcommand{\part{}}[1]{\textbf{\large Part \Alph{partCounter}}\stepcounter{partCounter}\\}
\renewcommand{\part}{
    \vspace{10pt}
    \textbf{(\alph{partCounter})}
    \stepcounter{partCounter}
}

% Alias for the Solution section header
\newcommand{\solution}[1][*]{
    \vspace{10pt}
    \ifx#1*
        \textbf{Rešitev (\alph{solCounter})}
        \stepcounter{solCounter}
    \else
        \textbf{Rešitev}
    \fi
}

\newcommand{\example}{
    \vspace{10pt}
    \textbf{Primer:}\\
}

\renewcommand{\d}{\;\mathrm d}
% \newcommand{\vphi}{\phi}
\renewcommand{\phi}{\varphi}
\newcommand{\eps}{\varepsilon}
\renewcommand{\hat}{\widehat}
\renewcommand{\tilde}{\widetilde}
\renewcommand{\bar}{\overline}
\newcommand{\subs}{\subseteq}
\newcommand{\nin}{\not\in}

\newcommand{\p}[1]{\left( {#1} \right)}
\newcommand{\set}[2]{\left\{ #1 \mid #2 \right\}}
\newcommand{\newfrac}[2]{{}^{#1}\!/_{\!#2}}
\newcommand{\quot}[2]{\newfrac{#1}{#2}}
\DeclareMathOperator{\im}{im}
\DeclareMathOperator{\coker}{coker}
\DeclareMathOperator{\coim}{coim}
\DeclareMathOperator{\id}{id}
\newcommand{\mb}[1]{\mathbold{#1}}
\newcommand{\mf}[1]{\mathfrak{#1}}
\newcommand{\mc}[1]{\mathcal{#1}}
\newcommand{\cc}{\complement}

\newcommand{\bin}[2]{\mathrm{Bin}{\p{#1, #2}}}

\newcommand{\for}[2]{\forall#1.\;#2}
\newcommand{\exist}[2]{\exists\;#1\smallni:\;#2}
\newcommand{\existi}[2]{\exists!\;#1\smallni:\;#2}

\renewcommand{\check}{ \(\checkmark\)}

\makeatletter
\newcommand{\oset}[3][0ex]{%
  \mathrel{\mathop{#3}\limits^{
    \vbox to#1{\kern-2\ex@
    \hbox{$\scriptstyle#2$}\vss}}}}
\makeatother


\newcommand{\hmwkTitle}{Projektna naloga}
\newcommand{\hmwkDueDate}{9.~2022}
\newcommand{\hmwkClass}{Statistika}
\newcommand{\hmwkClassTime}{}
\newcommand{\hmwkClassInstructor}{}
\newcommand{\hmwkAuthorName}{\textbf{Strah},~27181100}

\begin{document}

\maketitle

\pagebreak

\begin{homeworkProblem}
    V datoteki \texttt{Kibergrad} se nahajajo informacije o \(43.886\) družinah, ki stanujejo v mestu \emph{Kibergrad}. Mesto ima štiri četrti: v severni četrti stanuje \(10.149\) družin, v vzhodni \(10.390\), v južni \(13.457\) in v zahodni \(9.890\). Za vsako družino so zabeleženi naslednji podatki (ne boste potrebovali vseh):
    ˆ\begin{itemize}
        \item Tip družine (od \(1\) do \(3\))
              ˆ\item Število članov družine
              ˆ\item Število otrok v družini
              ˆ\item Skupni dohodek družine
              % \itemˆ Četrt, v kateri stanuje družina:
        \item Četrt, v kateri stanuje družina:
              \begin{enumerate}[1:]
                  \item Severna
                  \item Vzhodna
                  \item Južna
                  \item Zahodna
              \end{enumerate}
        \item Stopnja izobrazbe vodje gospodinjstva (od \(31\) do \(46\))
    \end{itemize}
    Iz vsake četrti vzemite enostavni slučajni vzorec velikosti \(100\).

    \part Primerjajte dohodke po četrtih, tako da narišete vzporedne škatle z brki (glejte razdelek 10.6 v knjigi). Je videti, da so določenih četrtih dohodki višji?

    \part Iz severne četrti vzemite še štiri enostavne slučajne vzorce velikosti \(100\). Za vseh pet vzorcev iz severne četrti spet narišite vzporedne škatle z brki. Komentirajte!

    \part Za celotni Kibergrad izračunajte varianco dohodka, pojasnjeno s četrtmi, in preostalo (rezidualno) varianco. Kako se to ujema z opažanji od prej?
\end{homeworkProblem}

\newpage

\begin{homeworkProblem}
    % \vspace{-1.5em}
    \parbox[c]{.66\textwidth}{
        Pri najlonskih palicah so preizkušali lomljivost (Bennett in Franklin, 1954). V podobnih okoliščinah so ulili \(280\) palic in vsako od njih preizkusili na petih mestih. Rezultati poskusa so prikazani v tabeli na desni.
        Če ima palica enakomerno strukturo, bi moralo biti število mest, na katerih se je zlomila, porazdeljeno binomsko \(\bin{5}{p}\) za določen neznan \(p\). To naj bo naša osnovna ničelna domneva. Privzamemo tudi, da so palice med seboj neodvisne.
    }\hspace{2em}
    \begin{tabular}{|c|c|}
        \hline
        št. lomov & št. palic \\
        \hline
        \(0\)     & \(157\)   \\
        \(1\)     & \( 69\)   \\
        \(2\)     & \( 35\)   \\
        \(3\)     & \( 17\)   \\
        \(4\)     & \(  1\)   \\
        \(5\)     & \(  1\)   \\
        \hline
    \end{tabular}

    \part Ob predpostavki osnovne ničelne domneve ocenite \(p\) po metodi največjega verjetja.

    \part Združite zadnje tri vrednosti in s posplošenim Pearsonovim preizkusom hi kvadrat preizkusite osnovno ničelno domnevo proti alternativni domnevi, da ima
    število lomov katero drugo porazdelitev (glejte razdelek 9.5 v knjigi). Še vedno
    privzamemo, da ima število lomov na vseh palicah enako porazdelitev.

    \part Za \( i ∈ \{1, …, n\} \) naj bodo dana neodvisna opažanja \(Xᵢ ~ \bin{mᵢ}{pᵢ}\), kjer
    so parametri \(mᵢ\) znani, parametri \(pᵢ\) pa neznani. Razvijte preizkus na podlagi
    razmerja verjetij, ki bo preizkusil ničelno domnevo, da so vsi parametri \(pᵢ\) enaki,
    proti alternativni domnevi, da temu ni tako.

    \part Uporabite preizkus iz prejšnje točke na danih podatkih, vedite pa, da Wilksovega izreka ne morete uporabiti (premislite, zakaj pogoji niso izpolnjeni).
    Namesto tega uporabite metodo \emph{bootstrap}: simulirajte \(10.000\) vrednosti preizkusne statistike pri ničelni domnevi, pri čemer za \(p\) vzemite oceno iz točke \textbf{(a)}. Nato poglejte, koliko teh vrednosti presega vrednost preizkusne statistike, izračunano na konkretnih podatkih. Na podlagi tega ustrezno sklepajte, kaj storiti z ničelno domnevo.

\end{homeworkProblem}

\newpage

\begin{homeworkProblem}
    V neki raziskavi:
    \[\text{\url{http://www.statsci.org/data/oz/ms212.html}}\]
    so študentom merili pulz. Vsakemu študentu so pulz izmerili dvakrat.
    Določeni so imeli med obema meritvama fizično obremenitev (tek na mestu), določeni ne.
    Podatki so zbrani v tabeli \texttt{Pulz}, pri čemer imajo stolpci naslednje pomene:\\
    \begin{tabular}{l l}
        \texttt{VISINA}      & telesna višina                                                        \\
        \texttt{TEZA}        & telesna teža                                                          \\
        \texttt{STAROST}     & starost v letih                                                       \\
        \texttt{SPOL}        & \(1=\)moški, \(2=\)ženski                                             \\
        \texttt{KADI}        & \(1=\)kadilec, \(2=\)nekadilec                                        \\
        \texttt{ALKOHOL}     & \(1=\)pije, \(2=\)ne pije                                             \\
        \texttt{VADBA}       & \(1=\)vadi veliko, \(2=\)vadi zmerno, \(3=\)vadi malo ali pa sploh ne \\
        \texttt{OBREMENITEV} & \(1=\)obremenitev, \(2=\)brez obremenitve                             \\
        \texttt{PULZ1}       & prva meritev pulza                                                    \\
        \texttt{PULZ2}       & druga meritev pulza                                                   \\
        \texttt{LETO}        & leto meritve \((1993-1998)\)
    \end{tabular}

    \part Preizkusite, ali to, ali so bili študenti deležni obremenitve, vpliva na spremembo
    pulza.

    \part Se zdi, da so določeni študenti, ki so bili določeni za obremenitev, morda
    goljufali in sploh niso tekli? Ilustrirajte s primernim grafičnim prikazom.

    \part Pri študentih, ki so bili deležni obremenitve, preizkusite, ali vadba vpliva na
    spremembo pulza.
\end{homeworkProblem}

\end{document}
