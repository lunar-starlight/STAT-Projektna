% LTeX: enabled=false
% Kodiranje in podpora slovenščini
\usepackage[T1]{fontenc}        % kodiranje znakov v .pdf
\usepackage[slovene]{babel}

\usepackage{fontspec}
\usepackage{lualatex-math}
\usepackage{unicode-math}

% \setmainfont{TeX Gyre Pagella}
% \setmathfont{TeX Gyre Pagella Math}
\setmathfont{Latin Modern Math}
\setmathfont{Asana Math}[range={scr}]
\setmathfont{STIX Two Math-Regular}[range={bb}]
% \setmathfont{Asana Math}[range={"007B,"007D}]  % {}
\setmathfont{Asana Math}[range={8709, \setminus}]  % U+2205, emptyset


% \usepackage{concrete}           % pisave po okusu Donalda Knutha

% Ta paket potrebujemo, ker amsart spreminja male črke v velike v naslovu,
% in tega ne zna pravilno delati s šumniki. Paket textcase ta problem odpravi.
% Glej: https://tex.stackexchange.com/questions/211305/problem-with-greek-title
\usepackage{textcase}

% Paketi za matematiko

\undef\eth
\undef\digamma
\undef\backepsilon
\usepackage{amsmath}  % razna okolja za poravnane enačbe ipd.
\usepackage{amsthm}   % definicije okolij za izreke, definicije, ...
% \usepackage{xypic}    % paket za diagrame

% \usepackage{mathtools}


\usepackage{fancyhdr}
\usepackage{extramarks}
\usepackage{enumerate}
\usepackage{tikz}
\usetikzlibrary{babel}
% \usepackage[plain]{algorithm}
% \usepackage{algpseudocode}
% \usepackage{listings}
\usepackage{minted}
\usepackage{hyperref}

\usetikzlibrary{cd,positioning}

%
% Basic Document Settings
%

\topmargin=-0.45in
\evensidemargin=0in
\oddsidemargin=0in
\textwidth=6.5in
\textheight=9.0in
\headsep=0.25in
\headheight=12.5pt

\linespread{1.1}

\pagestyle{fancy}
\lhead{\hmwkAuthorName}
\chead{\hmwkClass:\ \hmwkTitle}
\rhead{\firstxmark}
\lfoot{\lastxmark}
\cfoot{\thepage}

\renewcommand\headrulewidth{0.4pt}
\renewcommand\footrulewidth{0.4pt}

\setlength\parindent{0pt}

%
% Create Problem Sections
%

\newcommand{\enterProblemHeader}[1]{
    \nobreak\extramarks{}{Naloga \arabic{#1} se nadaljuje na naslednji strani\ldots}\nobreak{}
    \nobreak\extramarks{Naloga \arabic{#1} (nadaljevanje)}{Naloga \arabic{#1} se nadaljuje na naslednji strani\ldots}\nobreak{}
}

\newcommand{\exitProblemHeader}[1]{
    \nobreak\extramarks{Naloga \arabic{#1} (nadaljevanje)}{Naloga \arabic{#1} se nadaljuje na naslednji strani\ldots}\nobreak{}
    \stepcounter{#1}
    \nobreak\extramarks{Naloga \arabic{#1}}{}\nobreak{}
}

\setcounter{secnumdepth}{0}
\newcounter{partCounter}
\newcounter{solCounter}
\newcounter{homeworkProblemCounter}
\setcounter{homeworkProblemCounter}{1}
\nobreak\extramarks{Naloga \arabic{homeworkProblemCounter}}{}\nobreak{}

%
% Homework Problem Environment
%
% This environment takes an optional argument. When given, it will adjust the
% problem counter. This is useful for when the problems given for your
% assignment aren't sequential. See the last 3 problems of this template for an
% example.
%
\newenvironment{homeworkProblem}[1][-1]{
    \ifnum#1>0
        \setcounter{homeworkProblemCounter}{#1}
    \fi
    \section{Naloga \arabic{homeworkProblemCounter}}
    \setcounter{partCounter}{1}
    \setcounter{solCounter}{1}
    \enterProblemHeader{homeworkProblemCounter}
}{
    \exitProblemHeader{homeworkProblemCounter}
}


%
% Title Page
%

\title{
    \vspace{2in}
    \textmd{\textbf{\hmwkClass:\ \hmwkTitle}}\\
    \normalsize\vspace{0.1in}\small{Rok\ oddaje\ \hmwkDueDate}\\
    % \vspace{0.1in}\large{\textit{\hmwkClassInstructor\ \hmwkClassTime}}
    \vspace{3in}
}

\author{\hmwkAuthorName}
\date{}

% \renewcommand{\part{}}[1]{\textbf{\large Part \Alph{partCounter}}\stepcounter{partCounter}\\}
\renewcommand{\part}{
    \vspace{10pt}
    \textbf{(\alph{partCounter})}
    \stepcounter{partCounter}
}

% Alias for the Solution section header
\newcommand{\solution}[1][*]{
    \vspace{10pt}
    \ifx#1*
        \textbf{Rešitev (\alph{solCounter})}
        \stepcounter{solCounter}
    \else
        \textbf{Rešitev}
    \fi
}

\newcommand{\example}{
    \vspace{10pt}
    \textbf{Primer:}\\
}

\renewcommand{\d}{\;\mathrm d}
% \newcommand{\vphi}{\phi}
\renewcommand{\phi}{\varphi}
\newcommand{\eps}{\varepsilon}
\renewcommand{\hat}{\widehat}
\renewcommand{\tilde}{\widetilde}
\renewcommand{\bar}{\overline}
\newcommand{\subs}{\subseteq}
\newcommand{\nin}{\not\in}

\newcommand{\p}[1]{\left( {#1} \right)}
\newcommand{\set}[2]{\left\{ #1 \mid #2 \right\}}
\newcommand{\newfrac}[2]{{}^{#1}\!/_{\!#2}}
\newcommand{\quot}[2]{\newfrac{#1}{#2}}
\DeclareMathOperator{\im}{im}
\DeclareMathOperator{\coker}{coker}
\DeclareMathOperator{\coim}{coim}
\DeclareMathOperator{\id}{id}
\newcommand{\mb}[1]{\mathbold{#1}}
\newcommand{\mf}[1]{\mathfrak{#1}}
\newcommand{\mc}[1]{\mathcal{#1}}
\newcommand{\cc}{\complement}

\newcommand{\bin}[2]{\mathrm{Bin}{\p{#1, #2}}}

\newcommand{\for}[2]{\forall#1.\;#2}
\newcommand{\exist}[2]{\exists\;#1\smallni:\;#2}
\newcommand{\existi}[2]{\exists!\;#1\smallni:\;#2}

\renewcommand{\check}{ \(\checkmark\)}

\makeatletter
\newcommand{\oset}[3][0ex]{%
  \mathrel{\mathop{#3}\limits^{
    \vbox to#1{\kern-2\ex@
    \hbox{$\scriptstyle#2$}\vss}}}}
\makeatother
